\documentclass[11pt,a4paper,twoside,draft]{memoir}
\usepackage[final,protrusion,babel]{microtype}
\usepackage[utf8]{inputenc}
\usepackage[T1]{fontenc}
\usepackage[english]{babel}
\usepackage[style=english]{csquotes}
\usepackage{xspace}
\usepackage[final]{graphicx}
\usepackage[textsize=tiny]{todonotes}
\usepackage[final,hyperfootnotes=false]{hyperref}

% \usepackage{placeins}           % \FloatBarrier
%\usepackage{float}             % H float placement
\usepackage[originalcommands]{ragged2e} % improved ragged paragraphs in abstract
\usepackage{eso-pic}

\usepackage{amsmath}
\usepackage{amssymb}
%\usepackage{pifont}
\usepackage{fourier-orns}

\usepackage{typographic_setup}
\usepackage[final]{mylistings}
\usepackage{jlcode}

%-------------------------------------------------------------------------------
% TITLING
%
% \newcommand{\upinitial}[2]{{\textup{#1}\kern#2}}

% \title{Graph Tracking in Dynamic Probabilistic Programs via Source Transformations.} 
% \subtitle{\upinitial{A}{0pt} \upinitial{C}{-0.8pt}ase %
%   \upinitial{S}{-0.8pt}tudy with \upinitial{A}{-0.7pt}ction \upinitial{S}{-1.4pt}ystems}

% \author{Philipp Gabler}

% \date{}

\hypersetup{
  pdfinfo={
    Author={Philipp Gabler},
    Title={Graph Tracking in Dynamic Probabilistic Programs via Source Transformations.}
  },
  % colorlinks=false
}

%-------------------------------------------------------------------------------
% BIBLATEX
%

\usepackage[%
  backend=biber,
  citestyle=alphabetic,
  style=alphabetic,
  sortcites=true,
  sorting=nyt,
  %natbib=true,
  firstinits=true,
  %url=false,
  isbn=false,
  date=iso8601,
  urldate=iso8601
]{biblatex}
\DeclareNameAlias{default}{last-first}

% only capitalize real titles
% http://tex.stackexchange.com/a/22981/46356
\DeclareFieldFormat{sentencecase}{\MakeSentenceCase{#1}}
\renewbibmacro*{title}{%
  \ifthenelse{\iffieldundef{title}\AND\iffieldundef{subtitle}}
    {}
    {\printtext[title]{%
        \printfield[sentencecase]{title}%
        \setunit{\subtitlepunct}%
        \printfield[sentencecase]{subtitle}}
      \newunit}%
  \printfield{titleaddon}}

\defbibheading{memoir}[\bibname]{\chapter*{#1}}
\setcounter{biburllcpenalty}{7000}
\setcounter{biburlucpenalty}{8000}

\AtEveryBibitem{\clearlist{language}} % clears language
\renewcommand*\bibnamedash{\rule[0.48ex]{3em}{0.14ex}\space}
\renewcommand*{\multinamedelim}{;\space}
\renewcommand*{\finalnamedelim}{;\space}

\addbibresource{document.bib}

\usepackage{breakurl}
%\hypersetup{breaklinks=false}

%-------------------------------------------------------------------------------
% OTHER SETTINGS
%

% listings
% \lstset{
%   gobble=1,
%   language=Julia,
%   style=blackwhite
% }

\AtBeginDocument{%
  \lstMakeShortInline[style={inline}]|
}

% toc
\setlength{\cftbeforechapterskip}{1ex} % decrease space between sections

% floats and epigraphs
\newlength{\capwidth}
\addtolength{\capwidth}{\textwidth}
\addtolength{\capwidth}{-4ex}
\captionwidth{\capwidth}
\captionstyle[\raggedright]{}

\setlength{\epigraphwidth}{\textwidth}
\setlength{\epigraphrule}{0pt}

\renewcommand{\textfloatsep}{\baselineskip}

\setcounter{topnumber}{1}       % allow only one float per page

% no need for colors here...
% \colorlet{textred}{darkgray!80}
% \colorlet{textblue}{darkgray!80}

\newcommand{\lstlistingautorefname}{Listing}

% csquotes
% redefine spacing above/below; hacking original latex definition from 
% http://mirrors.ctan.org/macros/latex/base/classes.dtx
\makeatletter
\newenvironment{myblockquote}
               {\vspace{-1em}\list{}{\listparindent 1.5em%
                        \itemindent    \listparindent
                        \rightmargin   \leftmargin
                        \parsep        \z@ \@plus\p@}%
                \item\relax}
               {\endlist\vspace{-0.5\baselineskip}}
\makeatother
\SetBlockEnvironment{myblockquote}


% TIKZ SETUP
\usepackage{tikz}
% \usepackage{rotating}

%-------------------------------------------------------------------------------
% DOCUMENT MACROS
%
\newcommand{\ie}{i.e.\xspace}
\newcommand{\eg}{e.g.\xspace}
\newcommand{\cf}{cf.\xspace}
\newcommand{\margintodo}[1]{\todo[noline, size=\tiny]{#1}}
\newcommand{\CC}{{C\nolinebreak[4]\hspace{-.05em}\raisebox{.22ex}{\small\textbf{++}}}}
%\newcommand{\autosubref}[1]{{\hyperref[#1]{Subsection~\ref*{#1}~--~\nameref*{#1}}}}
\newcommand{\autosubref}[1]{\autoref{#1}}
\newcommand{\aref}[1]{\hyperref[#1]{Appendix~\ref{#1}}}
\newcommand{\mathlst}[1]{\text{\lstinline|1|}}


%-------------------------------------------------------------------------------
% DOCUMENT
%-------------------------------------------------------------------------------

\begin{document}
\pagestyle{simple}
\chapterstyle{hangnum}
\frontmatter

%%%% Time-stamp: <2015-04-05 11:32:08 vk>
%% ========================================================================
%%%% Disclaimer
%% ========================================================================
%%
%% created by
%%
%%      Stefan Kroboth and Karl Voit
%%
%% this title page fulfills the requirements of the corporate design
%% of Graz University of Technology (correct placement of logo)

\begin{titlingpage}

%\large  %% basic font size of the titlepage

%% placing the TU Graz logo exactly as Corporate Design demands:
%%   40mm of Logo (here: 2mm margin in picture!)
%%    8mm from top and from right
%% Source: http://portal.tugraz.at/portal/page/portal/TU_Graz/Services/BDR/Oeffentlichkeitsarbeit/CD/Logo%20Anwendungsrichtlinien
\AddToShipoutPicture*{%
  \AtPageUpperLeft{%
    \hspace{\paperwidth}%
    \raisebox{-19mm}{%\baselineskip}{%
     \makebox[-4mm][r]{\includegraphics[width=42mm]{figures/TU_Graz_Logo.pdf}}
}}}%

\begin{center}
~
\vfill\vfill\vfill

\sffamily

Philipp Gabler, BSc

\vfill

{\LARGE\bfseries sldfk sldkf}

{\large\bfseries sldkf sldf}

\vfill\vfill\vfill\vfill

{\normalsize\bfseries Masterarbeit}\\
\vfill
zur Erlangung des akademischen Grades\\
{Master of Science}\\
\vfill
eingereicht an der\\
{\normalsize\bfseries Technischen Universität Graz}

\vfill\vfill\vfill
\vfill\vfill\vfill

Betreuer\\
Univ.-Prof. Dipl.-Ing. Dr.\thinspace{}mont. Franz Pernkopf\\
\vfill
Mitbetreuer\\
Martin Trapp\\
\vfill
\vfill
Institut für Signalverarbeitung und Sprachkommunikation\\

\vfill

Fakultät für Informatik und Biomedizinische Technik

\vfill\vfill\vfill

{\scriptsize Graz, XXXX~2020}

\end{center}
\end{titlingpage}

%% end of title page
%%% Local Variables:
%%% mode: latex
%%% TeX-master: "main"
%%% End:

\movetooddpage

%%%% Time-stamp: <2017-02-14 16:01:12 vk>
%% ========================================================================
%%%% Disclaimer
%% ========================================================================
%%
%% created by
%%
%%      Karl Voit, and Matthias Wölbitsch
%%
%%
%% code for the date and signature fields adapted from
%% http://tex.stackexchange.com/a/20562


\newcommand{\textfield}[2]{
  \vbox{
    \hsize=#1\kern3cm\hrule\kern1ex
    \hbox to \hsize{\strut\hfil\footnotesize#2\hfil}
  }
}


\section*{Affidavit}
I declare that I have authored this thesis independently, that I have
not used other than the declared sources/resources, and that I have
explicitly indicated all material which has been quoted either
literally or by content from the sources used. The text document
uploaded to \abbrev{TUGRAZ}online is identical to the present master‘s
thesis.

\hbox to \hsize{\textfield{4cm}{Date}\hfil\hfil\textfield{7cm}{Signature}}



%%% Local Variables:
%%% mode: latex
%%% TeX-master: "main"
%%% End:

\movetooddpage


\begin{adjustwidth}{\absleftindent}{\absrightindent}
  \phantomsection
  % \addcontentsline{toc}{chapter}{Licenses}
  \label{license}
  \currentpdfbookmark{License}{license}

  \abstracttextfont
  \vspace*{\stretch{1}}
  \begin{center}
    This work is licensed under a \\ \href{http://creativecommons.org/licenses/by-sa/4.0/}{Creative
      Commons Attribution-ShareAlike 4.0 International License}.
  \end{center}
  \begin{center}
    \includegraphics[scale=1]{figures/by-sa}
  \end{center}
  \vspace{\stretch{2}}
  \begin{center}
    All code samples, unless otherwise noted or cited from other sources, \\ are also available under an
    \href{http://opensource.org/licenses/MIT}{\abbrev{MIT} license}:
  \end{center}
  \vspace*{-1ex}
  \begin{ttfamily}
    \setlength{\parskip}{12pt}
    \setlength{\parindent}{0pt}
    The MIT License (MIT)

    Copyright (c) 2020 Philipp Gabler

    Permission is hereby granted, free of charge, to any person obtaining a copy of this software
    and associated documentation files (the "Software"), to deal in the Software without
    restriction, including without limitation the rights to use, copy, modify, merge, publish,
    distribute, sublicense, and/or sell copies of the Software, and to permit persons to whom the
    Software is furnished to do so, subject to the following conditions:

    The above copyright notice and this permission notice shall be included in all copies or
    substantial portions of the Software.

    THE SOFTWARE IS PROVIDED "AS IS", WITHOUT WARRANTY OF ANY KIND, EXPRESS OR IMPLIED, INCLUDING
    BUT NOT LIMITED TO THE WARRANTIES OF MERCHANTABILITY, FITNESS FOR A PARTICULAR PURPOSE AND
    NON\-IN\-FRINGE\-MENT. IN NO EVENT SHALL THE AUTHORS OR COPYRIGHT HOLDERS BE LIABLE FOR ANY
    CLAIM, DAMAGES OR OTHER LIABILITY, WHETHER IN AN ACTION OF CONTRACT, TORT OR OTHERWISE, ARISING
    FROM, OUT OF OR IN CONNECTION WITH THE SOFTWARE OR THE USE OR OTHER DEALINGS IN THE SOFTWARE.
  \end{ttfamily}
  
  \vspace{2ex}
  
  \begin{adjustwidth}{0.5\absleftindent}{0.5\absrightindent}
    \begin{Center}
      % A full |sbt| project containing many of the code samples can be found at
      % \url{https://github.com/phipsgabler/dsl-examples}. 

      The \LaTeX{} source of this document is available at\\
      \url{https://github.com/phipsgabler/master-thesis} \\
      or upon request from the author.\footnote{\texttt{pgabler@student.tugraz.at}}
    \end{Center}
  \end{adjustwidth}
  
  \vspace{\stretch{1}}

\end{adjustwidth}

%%%%%%%%%%%%%%%%%%%%%%%%%%%%%%%%%%%%%%%%%%%%%%%%%%%%%%%%%%%%%%%%%%%%%%%%%%%%%%%%
% ABSTRACT
\movetooddpage
\phantomsection
% \addcontentsline{toc}{chapter}{Abstract}
\label{abstract}
\currentpdfbookmark{Abstract}{abstract}
\begin{abstract}

\noindent
Alles sehr abstract hier.

\end{abstract}

\begingroup
\cleartorecto
\hypersetup{linkcolor=black, hyperindex=true}
\label{toc}
\currentpdfbookmark{Contents}{toc}
\tableofcontents*
\endgroup

\cleartorecto
\mainmatter


\chapter{Test}
Hello, there!

% \input{1-introduction.tex}

% \input{2-concepts.tex}

% \input{3-syntax.tex}

% \input{4-patterns.tex}

% \input{5-action_systems.tex}

% \input{6-actium.tex}

% \input{7-resumee.tex}


%-------------------------------------------------------------------------------
\clearpage
\vspace{2em}
\appendix


\begingroup

\microtypesetup{protrusion=false, expansion=false}
%\raggedright
\phantomsection
\addcontentsline{toc}{chapter}{Bibliography}
\printbibliography[heading=memoir]

\endgroup

%-------------------------------------------------------------------------------
\appendix

% \chapter{Example Programs}
% \label{sec:appendix}

% \section{SimpleRocket in Original Syntax}
% \label{sec:simplerocket_original}
% \lstinputlisting[style=fullpage,language={}]{code/SimpleRocket.acsys}


%-------------------------------------------------------------------------------
\cleartoverso
\thispagestyle{empty}
\renewcommand{\abstractname}{Colophon}
\begin{abstract}
  \noindent
  This document was typeset using the pdf\LaTeX{} typesetting system, with the memoir document
  class. The body text is set in 11\,pt Linux Libertine, enhanced by the microtype package. Other
  fonts include Biolinum and Inconsolata.
  % The drawings are typeset using
  % TikZ/PGF, and source code examples are formatted by the listings package.

  The document source has been written in Emacs with AUC\TeX{} mode, using TeXworks as \abbrev{PDF}
  viewer.
\end{abstract}

\end{document}

%%% Local Variables: 
%%% TeX-master: "main"
%%% mode: latex
%%% TeX-command-extra-options: "-shell-escape"
%%% End:
