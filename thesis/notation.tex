\chapter*{Notation}
\label{cha:notation}
\addcontentsline{toc}{chapter}{Notation}
% \currentpdfbookmark{Notation}{notation}

\begin{symbolicfootnotes}
  \begin{description}[style=nextline, leftmargin=4cm]
  \item[\(\Prob{\Theta \in A \given X = x}\)] Random variables and their realizations will usually be
    denoted by upper and lower case letters, respectively (with occasional exceptions for Greek
    variable names).  Sets are also named by uppercase letters.
  \item[{\(\Exp{X}, \Var[X]{f(X, Y)}\)}] Expectation and variance; if necessary, the variable with
    respect to which the moment is taken is indicated as a subscript.
  \item[\(\phi(x), f_{Z}(x)\)] Density functions are named using letters commonly used for functions,
    with an optional subscript indicating the random variable they belong to.  Densities always come
    with implied base measures depending on the type of the random variable.
  \item[\(\prob{x, y \given z}\)] The usual abuse of notation with the letter \enquote{p} standing for
    any density indicated by the names of the variables given to it is used when no confusion arises
    (in this case, \(f_{X,Y|Z}\) is implied).  A \(q\) may be used as well, mostly for proposal
    distributions or unnormalized densities.
  \item[{\(\Prob{X \in A} = P_{X}(A) = \int_{A} p_{X}(x) \dif\mu(x)\)}] A capital \(P\) with subscript
    is used for the probability measure associated with a random variable.
  \item[\(\Prob{X \in \dif x} = P_{X}(\dif x) = p_{X}(x) \dif\mu(x)\)] Differentials of this form
    are used to concisely express densities of random variables, i.e., Radon-Nikodym derivatives of
    the associated probability measure.  The example is equivalent to the statement
    \(\frac{\dif P_{X}}{\dif \mu} = p_{X}\).
  \item[{\(X_{i} \from \distr{Normal}(\mu, \sigma)\)}] The tilde notation for describing random
    variables is used throughout, often without explicitly specifying dependence or independence,
    where understood from context.  Named distributions that are not themselves random variables are
    spelled out in upright script.
  \item[{\(Y \from q(\cdot, X_{i-1})\)}] The same notation is used when a random variable is specified
    to be sampled from a given, possibly unnormalized, density.  In this context and elsewhere, the
    midpoint is employed to denote anonymous functions of one variable given by partial application.
  \item[{\(y \mapsto \prob{x \given y, z}\)}] Anonymous functions are distinguished from function
    evaluation; this is crucial to differentiate between probability densities and likelihoods, for
    example.
  \item[\(\int \prob{x} \dif x = 1\)] Integrals over the whole domain of a density or measure are
    written as indefinite integrals, where the usage is clear.
  \item[\(\indicator{z_n = k}\)] Indicator function.  The value is \(1\) when the predicate inside
    holds, and \(0\) otherwise.
  \item[{\([x, y, z] = \smash[b]{\left(\begin{smallmatrix}x\\y\\z\end{smallmatrix}\right)}\)}] For
    consistency with Julia code, vectors (arrays of rank \(1\)) are written in brackets, with
    elements separated by commas.  Thereby, the form written in a row denotes a column vector;
    actual row vectors are written as transposed column vectors.
  \item[{\(\kth{\Theta} = [\kth{\Theta}_1, \ldots, \kth{\Theta}_N]\)}] Superscript indices in
    parentheses are used for series or sequences of variables, and subscript indices for components of
    multivariate variables.
  \item[{\(z_{-i} = [z_{1}, \ldots, z_{i-1}, z_{i+1}, \ldots, z_{N}]\)}] Negative indices denote all
    components of a variable without the negated one.
  \item[{\(\broadcast{f}(x, 1) = [f(x_{1}, 1), \ldots, f(x_{N}, 1)]\)}] Function application with a
    period indicates vectorized application, as in Julia
    code\footnote{See~\protect\url{https://docs.julialang.org/en/v1/manual/functions/\#man-vectorized-1}}:
    the function is applied over all elements of the input arrays individually, whereby arrays of
    lower rank or scalars are \enquote{broadcasted} along dimensions as necessary.
  \item[{\(\Dif F(x, y) = (\Delta_1, \Delta_2) \mapsto \partial_1F(x, y)\,\Delta_1 + \partial_2F(x,
      y)\,\Delta_2\)}] Derivatives are written using a capital \(\Dif\) for the a total derivative
    operator, and \(\partial_i\) for the conventional partial derivatives with respect to the
    \(i\)-th argument.
  \item[{\jlinlfont f(x) = rand(x)}] Julia code (including identifiers mention in the text) is
    always typeset in typewriter font.
  \item[{\irtrackerjl{}}] Julia packages are set in the same font as Julia code.  In the electronic
    version, a hyperlink to their source code on Github is automatically added.
  \end{description}
\end{symbolicfootnotes}

%%% Local Variables: 
%%% TeX-master: "main"
%%% End: