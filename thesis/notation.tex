\chapter*{Notation}
\label{cha:notation}
\addcontentsline{toc}{chapter}{Notation}
% \currentpdfbookmark{Notation}{notation}

\begin{description}[style=nextline, leftmargin=4cm]
\item[\(\Prob{\Theta \in A \given X = x}\)] Random variables and their realizations will usually be
  denoted with upper and lower case letters, respectively
  (with some exceptions for Greek variable names).  Sets
  are written with uppercase letters.
\item[{\(\Exp{X}, \Var[X]{f(X, Y)}\)}] Expectation and variance; if necessary, the variable with
  respect to which the moment is taken is indicated.
\item[\(\phi(x), f_{Z}(x)\)] Density functions are named using letters commonly used for functions,
  with an optional subscript indicating the random variable they belong to.  Densities always come
  with implied base measures depending on the type of the random variable.
\item[\(\prob{x, y \given z}\)] The usual abuse of notation with the letter \enquote{p} standing for
  any density indicated by the names of the variables given to it is used when no confusion arises
  (in this case, \(f_{X,Y|Z}\) is implied).
\item[{\(y \mapsto \prob{x \given y, z}\)}] Anonymous functions are distinguised from function
  evaluation; this is crucial to differentiate between probability densities and likelihoods, for
  example.
\item[\(\int \prob{x} \dif x = 1\)] Integrals over the whole domain of a density are written as
  indefinite integrals, where the usage is clear.
\item[{\([x, y, z] = \smash[b]{\begin{bsmallmatrix}x\\y\\z\end{bsmallmatrix}}\)}] For consistency
  with Julia code, vectors (arrays of rank \(1\)) are written in brackets.  Thereby, the form
  written in a row denotes a column vector; actual row vectors are written as transposed column
  vectors.
\item[{\(\kth{\Theta} = [\kth{\Theta}_1, \ldots, \kth{\Theta}_N]\)}] Superscript indices in
  parentheses are used for series or sequences of variables, and subscript indices for components of
  multivariate variables.
\item[{\(z_{-i} = [z_{1}, \ldots, z_{i-1}, z_{i+1}, \ldots, z_{N}]\)}] Negative indices denote all
  components of a variable without the negated one.
\item[{\(\broadcast{f}(x, 1) = [f(x_{1}, 1), \ldots, f(x_{N}, 1)]\)}] Functions with a period
  indicate vectorized application, as in Julia
  code\footnote{See~\protect\url{https://docs.julialang.org/en/v1/manual/functions/\#man-vectorized-1}}:
  the function is applied over all elements of the input arrays individually, whereby arrays of
  lower rank or scalars are \enquote{broadcasted} along dimensions as necessary.
\item[Julia code] Julia code (including identifiers mention in the text) is always typeset in
  \jlinl{typewriter font}.
\end{description}

%%% Local Variables: 
%%% TeX-master: "main"
%%% End: