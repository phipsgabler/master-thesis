\chapter*{Notation}
\label{cha:notation}
\addcontentsline{toc}{chapter}{Notation}
% \currentpdfbookmark{Notation}{notation}

\begin{center}
  \renewcommand{\arraystretch}{1.5}
  \begin{tabular}{lp{0.6\textwidth}}
    \(\Prob{\Theta \in A \given X = x}\) & Random variables and their realizations will usually be
                                           denoted with upper and lower case letters, respectively
                                           (with some exceptions for Greek variable names).  Sets
                                           are written with uppercase letters.\\
    \(\Exp{X}\), \(\Var[X]{Y}\) & Expectation and variance; if necessary, the variable with respect
                                  to which the moment is taken is indicated. \\
    \(f(x)\), \(\phi_Z(x)\) & Density function are named using letters commonly used for functions,
                              with an optional subscript indicating the random variable they belong
                              to. \\
    \(\prob{x, y \given z}\) & The usual abuse of notation with the letter \enquote{p} standing for
                               any density indicated by the names of the variables given to it is
                               used as well (in this case, \(f_{X,Y|Z}\) is implied). \\
    \(\int \prob{x} \dif x = 1\) & Integrals over the whole domain of a density are written as
                                   indefinite integrals, where the usage is clear.
  \end{tabular}
\end{center}

%%% Local Variables: 
%%% TeX-master: "main"
%%% End: