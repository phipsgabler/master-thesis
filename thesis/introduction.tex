\chapter{Introduction}
\label{cha:introduction}

This chapter gives an overview over the scope of the thesis and existing approaches in the
literature.  It is based on \textcite{gabler2019graph}, which presents a preliminary version of this
work.

\newthought{In machine learning}, several methods work on computation graphs of programs that
represent mathematical expressions.  One example are is automatic differentiation (AD), which
derives new expressions from an expression that usually represents a loss function, to calculate its
gradient \parencite{griewank2008evaluating, gebremedhin2020introduction}.  AD is a special case of
more general message passing algorithms
\parencite{minka2005divergence,ruozzi2011message,minka2019automatic}, which all require a graph as
basic data structure for the operation they perform: there, values are passed between nodes,
representing variables in a domain that depend on each other.  Moreover, in other fields, such as
program analysis or program transformation
\parencite[cf.][]{muchnick1997advanced,singer2018static,aho1986compilers}, the same requirements
might occur through the need to derive abstract graphs of program flow from a given program.

There are several options how to extract the computation graph in question, many of which are
already established in the AD community (see \textcite{baydin2018automatic} for a survey on AD
methods).  For one, graphs can be required to be written out explicitly by the user, by providing a
library to build graphs \enquote{by hand} (e.g. \textcite{chewxy2020gorgonia,jia2014caffe}) in a
more low-level application programming interface (API).  Alternatively, there are higher-level APIs
like PyTorch \parencite{paszke2017automatic} or AutoGrad \parencite{maclaurin2015autograd}, where
the graphs are defined implicitely from a forward program written in more declarative style
(TensorFlow \parencite{abadi2015tensorflow} is somewhere between these).  Such APIs are called
\emph{operator overloading} in AD terminology, because they extend existing operations to
additionally track the computation graph at run-time on so-called tapes or Wengert lists
\parencite{bartholomew-biggs2000automatic}.  This kind of tracking is dynamic, in the sense that a
new tape is recorded for every execution.  However, being implemented on a library level, it usually
requires the programmer to use non-native constructs instead of language primitives, leading to
cognitive overhead, while it also makes the applicability limited to library functions and not
easily extendable.  This notably happens for control statements, which can rarely be
\enquote{overloaded}.  Furthermore, there are additional run-time costs due to the separate
interpretation of derivatives stored on the tape.

Alternatively, an implementation can allow the user to write out computations as a \enquote{normal}
program in an existing programming language (or possibly a restricted subset of it), and use program
transformation techniques to extract graphs from the input program.  Such meta-programs, known as
\emph{source transformations}, can in turn operate on plain source code (cf. Tapenade,
\textcite{tapenadedevelopers2019tapenade}), or on another, more abstracted notion used by the
programming language infrastructure, like the abstract syntax tree (AST), or an intermediate
representation (IR).  They operate on the syntactic structure of the whole program, during or before
compilation.  Unlike in operator overloading, it is hence possible to inspect and exploit control
structures directly. This can lead to more efficient results, compared to operator overloading,
since the transformation is done only once per program and eligible for compiler optimizations.
Whereas the user is not restricted to the domain specific language provided by a library, and can
use regular language constructs, data structures, and custom functions rather freely, in this
approach, usually, no records of the actual execution paths are constructed explicitly.  Only purely
static information is used only at compile time, and cannot be accessed for further analysis or
transformation during execution.  (See section~\ref{sec:cg-ad} for a more in-depth treatment of
automatic differentiation techniques.)

For probabilitic programming languages, there exist mainly two paradigms for handling program
structure \parencite{vandemeent2018introduction}.  In the case of evaluation-based systems (e.g.,
\turingjl{} \parencite{ge2018turing}, \juliapackage{Gen.jl}'s dynamic interface
\parencite{cusumano-towner2020gen}, Church \parencite{goodman2012church}, Anglican
\parencite{wood2015new}, Pyro \parencite{bingham2018pyro}), no structure is extracted at all.  The
interaction between the system and user programs consists only of a sequence of messages (in an
abstract sense), indicating \enquote{events} that can be taken up by inference algorithms.  In
graph-based systems, an static representation of the model is first constructed and then passed to
the inference algorithm.  This representation can be close to probabilistic graphical models
(\juliapackage{ForneyLab.jl}, \parencite{cox2018forneylab}, Venture
\parencite{mansinghka2014venture}, PyMC3 \parencite{salvatier2016probabilistic}), symbolic
(\juliapackage{Gen.jl}'s static interface, \juliapackage{Soss.jl} \parencite{scherrer2019soss}), or
more compiler-oriented as in Stan \parencite{carpenter2017stan}, BUGS \parencite{lunn2000winbugs},
or JAGS \parencite{plummer2003jags}.

In a variety of modelled domains, though, the execution path of programs can drastically change at
each run.  Examples of this are implementations of models containing non-uniform data, such as parse
trees \parencite{socher2011parsing} or molecular graphs \parencite{bianucci2000application}, of
Bayesian nonparametric models \parencite{hjort2010bayesian}, or simply the occurrence of stochastic
control flow in any probabilistic model.  We call such models dynamic.  The lack of an explicit,
unique graph structure makes it impossible, or at least difficult, to apply source transformation
approaches on their implementation.  Operator overloading is the more direct way for supporting
dynamic models, since it automatically records a new tape for each input. In fact, many of the
already mentioned state-of-the-art machine learning libraries are based on dynamic graphs using
operator overloading in some form.

Reliance on operator overloading makes it impossible to take advantage of the benefits of source
transformations, such as utilizing information about the control flow, integrating with
optimizations at compile time, or exploiting the source model structure.  A source transformation
approach based on intermediate compiler representations has recently gained popularity in machine
learning, and promises to resolve this dilemma; see \textcite{bradbury2018jax,lattner2020mlir}.
While the main focus of these efforts has been optimization of linear algebra/tensor calculations
and automatic differentiation, other use cases start to emerge, for example automatic detection of
sparsity patterns \parencite{gowda2019sparsity}.

\newthought{In this thesis}, I present a novel variant of automatic extraction of computation graphs
suitable for static and dynamic models, using IR-based source transformation instead of operator
overloading.  Inspired by recent work on differential programming \parencite{innes2018don}, the
approach transforms the intermediate representation used by the compiler of the Julia programming
language.  This system can be used to dynamically track computation graphs of any Julia program,
including machine learning models and probabilistic programming systems, without having to
explicitly declare graph structures. The transformation is implemented as a custom part of the
compilation process.  Its result is passed back to the compiler, where it can be optimized
further. At run time, both data and control path are tracked alongside the original calculations, in
the form of a nested data structure.  This data structure contains all functions called during
execution, enriched by recorded control flow decisions and possibly meta-information that can be
used to analyse the execution. Thus, the system combines advantages of a source transformation with
a tape-based run-time approach.

The extracted graphs can be used for various purposes.  It is possible to implement automatic
differentiation on top of it, as well as other algorithms that can be formulated via message-passing
\parencite{minka2019automatic,minka2005divergence}, and methods that operate on run-time dependency
graphs, from simple debugging to concolic execution \parencite{sen2005cute,zeller2019concolic}.  As
a specific use case in the field of Bayesian inference, a dependency tracking system for a class of
models in the Julia-base probabilistic programming language \turingjl{} has been implemented.  On
top of this, Gibbs conditionals can automatically be derived for the models and used in compound
MCMC algorithms, similar to JAGS and BUGS.


%%%%%%%%%%%%%%%%%%%%%%%%%%%%%%%%%%%%%%%%%%%%%%%%%%%%%%%%%%%%%%%%%%%%%%%%%%%%%%%%%%%%%%%%%%%%%%%%%%%
\section{Related Work}
\label{sec:related-work}

The topic of this thesis crosses several disciplines~-- at least automatic differentiation, compiler
and programming language theory, and probabilistic programming.  Since these have not always worked
together, similar principles may be have been found or (re-)introduced in each of them.

Automatic differentiation has a long history, in which different styles became more or less
fashionable depending on the dominating use-case and available languages and infrastructure.
Traditionally, numerical code in Fortran or C was differentiated by whole-source transformation
systems like Tapenade \parencite{tapenadedevelopers2019tapenade}.  After phase of many operator
overloading systems that were driven by the rise of deep learning
\parencite{abadi2015tensorflow,paszke2017automatic,neubig2017dynet,tokui2015chainer}, compiler-based
approaches have regained popularity.  More recently, there have been efforts to add built-in
automatic differentiation to the Swift programming language in Swift for TensorFlow
\parencite{tensorflowdevelopers2018swift}, and work in Julia for \juliapackage{Zygote.jl}
\parencite{innes2018don}.  Both started to apply source transformation to the intermediate
representation of the compiler, which enables differentiating through complex control flow, custom
data types, and nested functions.  A similar approach to \juliapackage{Zygote.jl} is taken in Python
with Tangent \parencite{vanmerrienboer2018tangent}.

Generalizations of the kinds of analyses and transformations used in these systems can be found
under multiple terms in the compiler literature: data- and control-flow analysis, information
propagation, or abstract interpretation \parencite{muchnick1997advanced,singer2018static}.  There,
the program structure is always assumed to be known statically, though, as compilers fundamentally
are source transformers.  Closest to an evaluation-based analysis are concolic execution methods
\parencite{sen2005cute,zeller2019concolic}, in which a given program is \enquote{instrumented}
through additional instructions that, next to the concrete evaluation, track the execution in
symbolic form (hence the name).  The symbolic information can then be used in formal methods to, for
example, find sets of program input that ensure complete test coverage of all branches.  There
already exists a proof-of-concept implementation of a concolic fuzzer in Julia
\parencite{churavy2019vchuravy}, which applies the same kind IR-based transformations as
\juliapackage{Zygote.jl}.

These information propagation methods, most of which find recursive solutions to equations defined
over program structure, can in turn be seen as a form of message passing, under which not only a
variety of learning algorithms can be summarized \parencite{minka2005divergence}, but also automatic
differentiation \parencite{minka2019automatic} and gradient based optimization
\parencite{dauwels2005steepest}.  Other forms of abstract analysis exist for program optimization,
e.g., sparsity detection in numerical programming \parencite{gowda2019sparsity}, or the detection of
parts of probabilistic programs that need not to be reevaluation after input changes
\parencite{becker2020dynamic}.

Many implementations of these methods do not use the original form of the program, but a
syntactically simplified lowered form.  Such forms can be dependency graphs as used in compiler
theory, or the intermediate languages used by actual compiler implementations.  These can take
portable, language independent forms as in LLVM \parencite{llvmproject2019llvm}, or be special to a
particular compiler implementation, as in Julia \parencite{bezanson2017julia} or Swift
\parencite{apple2020swifta}.  As these two languages illustrate, there are often even multiple
layers of intermediate representations used in the same system.

Lately, special intermediate representations for machine learning applications have been introduced.
One of them is the machine learning intermediate representation (MLIR, \textcite{lattner2020mlir}),
with the purpose of forming a reusable mid-layer between programming languages and run-times,
featuring exchangeability between different machine learning frameworks and and
\enquote{accelerators} (pieces of hardware), while taking advantage of modern compiler technology
like LLVM.  Another one is Swift's intermediate representation (SIL) in the Swift for TensorFlow
project.  JAX \parencite{bradbury2018jax} plays a similar role for expression-graph based machine
learning systems, by tracing Python functions and compiling their graphs directly to optimized code.
This system can interact with XLA (\enquote{accelerated linear algebra},
\textcite{tensorflowdevelopers2020xla}), which allows to compile sequences of numerical Python
functions, which would otherwise be slow, to efficiently fused platform code.

As for the trade-off between transformation-based and evaluation-based implementations, several
hybrid graph tracking approaches between source transformation and graph tracking exist.  Among AD
systems, recent TensorFlow versions have introduced
AutoGraph\footnote{\url{https://www.tensorflow.org/api_docs/python/tf/autograph}, visited
  on~2020-10-26}, which rewrites regular Python functions to traced TensorFlow implementations by
replacing control flow statements with TensorFlow combinators.  Such functions still need to be
re-traced whenever a non-tensor input argument changes.  Its predecessor TensorFlow Fold
\parencite{looks2017deep} follows a similar, but more explicit style and provides many of these
combinators as \enquote{dynamic batching operators} to define static graphs emulating dynamic
operations.  In probabilistic programming, \enquote{dynamicity problem} can be approached in other
ways as well: a technique called \emph{stochastic memoization} is employed in the probabilistic
programming languages Church \parencite{goodman2012church} and Venture
\parencite{mansinghka2014venture} to produce what in the latter is called \enquote{probabilistic
  execution traces}, where multiple different traces are dynamically stored as alternative parallel
paths in the execution trace, with possible interconnections.  \juliapackage{Gen.jl}
\parencite{cusumano-towner2019gen,cusumano-towner2020gen} in Julia is defined over a single abstract
interface, for which two implementations are defined: a dynamic one, where dictionary-like traces
are recorded at runtime from general programs, and a static one, that converts a restricted,
combinator-based model syntax to a fixed graph structure through meta-programming.


%%% Local Variables: 
%%% TeX-master: "main"
%%% End: