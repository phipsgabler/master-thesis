\renewcommand{\abstractname}{Abstract}
\begin{abstract}
  \phantomsection
  \label{abstract}
  \currentpdfbookmark{Abstract}{abstract}
  
  \noindent Probabilistic programming is a means of describing stochastic models through the syntax
  of a programming language.  These models describe probability distributions as stochastic
  function.  Probabilistic programms distinguish themselves from normal programs by the possibility
  of being sampled from conditionally, with some of the internally variables fixed to observed
  values.  While probabilistic programming systems are often implemented as separate,
  domain-specific languages, they can also be embedded into \enquote{host} programming languages
  with sufficient syntactic flexibility.  The latter is advantageous if one wants to use regular
  general-purpose programming constructs or interact with other functionalities of the host
  language.

  This thesis first presents a general system for extracting rich computation graphs in the Julia
  programming language.  These graphs contain the whole recursive structure of a program in terms of
  executed instructions of the intermediate representation used by the compiler.  The system is
  flexible enough to be used for multiple purposes that require dynamic program analysis or abstract
  interpretation, such as automatic differentiation or dependency analysis.

  The second contribution is the application of the described graph tracking system to programs
  written for Turing, a probabilistic programming system implemented as an embedded language within
  Julia.  The functions describing a Turing model can be analyzed, and from them the dependency
  structure of involved random variables be extracted.  Given this structure, analytical Gibbs
  conditionals can be calculated and passed to Turing's inference mechanism, where they are used in
  Markov-Chain Monte Carlo samplers approximating the modelled distribution.
\end{abstract}

%%% Local Variables:
%%% mode: latex
%%% TeX-master: "main"
%%% End: