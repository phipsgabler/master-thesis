\chapter{Discussion}
\label{cha:discussion}

The history of this project forms a large arc from a general problem in \turingjl{}, over a
digression into compiler technology, back the the implementation of a proof of concept in the form
of a very specific inference method.  As we have seen, two separate pieces of software have emerged
from it: \irtrackerjl{} and \autogibbsjl{}.  The underlying issue~-- that \turingjl{} lacks a
structural representation of models~-- is not at all resolved by them.

The real difficulty is that dynamic models cannot be satisfactorily handled through static snapshots
(cf. Venture).

(\juliapackage{Cassette.jl} is a package very similar to \juliapackage{IRTools.jl})
\begin{quote}
  Using Cassette on code you wrote is a bit like shooting youself with a experimental mind control
  weapon, to force your hands to move like you knew how to fly a helicopter. Even if it works, you
  still had to learn to fly the helicopter in order to program the mind-control weapon to force
  yourself to act like you knew how to fly a helicopter.\footnote{Lyndon White, private
    communication on \protect\url{julialang.slack.com}.}
\end{quote}\todo{ask for permission}

In summary, I 

\section{Future Work}
\label{sec:future-work}

Bla.\footnote{These ideas have already been informally described by me online at
  \protect\url{https://github.com/phipsgabler/probability-ir}.}




%%% Local Variables: 
%%% TeX-master: "main"
%%% End: