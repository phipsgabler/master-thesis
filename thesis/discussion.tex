\chapter{Discussion}
\label{cha:discussion}

The history of this project forms a large arc, starting from a general problem in \turingjl{}, over
a digression into compiler technology and automatic differentiation, back the the implementation of
a proof of concept in the form of a very specific inference method.  As we have seen, two separate
pieces of software have emerged from it: \irtrackerjl{} and \autogibbsjl{}.  On its own, the latter
is indeed an improvement over the previous situation: Gibbs conditional samplers can be
significantly faster, thus allowing longer chains, than particle-based samplers, the go-to
instrument for discrete variables in \turingjl{} so far.  Already the addition of a \enquote{manual}
Gibbs conditional sampler in \turingjl{} allows to directly implement many models from the
literature, for which conditionals are often provided analytically.  Automatic derivation allows to
generalize this to a large class of models that have been found useful in other systems such as
JAGS.  The underlying issue~-- that \turingjl{} lacks a structural representation of models~-- is
not at all resolved by the implementation, unfortunately.  This makes \autogibbsjl{} not completely
satisfactory, since the recursion and branch tracking features of \irtrackerjl{} cannot be applied
in a useful way.

But the real difficulty is that dynamic models cannot be satisfactorily handled through
snapshot-like slices in the form of traces.  Systems trying to achieve this either become
restrictive in their expressibility, or very complex in some aspects, up to practical limitation
(see \textcite{mansinghka2014venture} and \textcite{goodman2012church}).  Furthermore, during
implementation of the work, the two main practical difficulties were matching of variable names
(e.g., subsuming \jlinl{x[1:10]} under \jlinl{x[1:3][2]}), and the correct handling of mutations
that shadow actual data dependencies (e.g., when one has an array \jlinl{x}, samples \jlinl{x[1]},
writes it to \jlinl{x} with \jlinl{setindex!}, and then uses \jlinl{getindex(x, i)} somewhere
downstream).  These cannot be really solved using external tooling, in the way that has been
demonstrated.  A more versatile dictionary structure for variable name keys could improve the
situation for variable names, but wouldn't solve all of the underlying issues.

There is also a fragility problem: Julia IR, while being publicly documented, is a rather internal
feature of the language, and may change between compiler versions.  The \juliapackage{IRTools.jl}
package provides a good mid-layer mitigating this, but still there's lot of reasons why a custom IR
would be nicer.  From the other side, also the structure of \dppljl{}'s model representations is not
stable, and actually an implementation detail that should not be relied on from the outside~--
especially not by an important feature such as dependency extraction.  In a certain sense, the whole
approach is misguided: why rely on external tracking for a framework that is really under ones own
control, using such heavy machinery as IR transformations?  This is illuminated by following very
telling comment about similar tendencies (\juliapackage{Cassette.jl} is a package very similar to
\juliapackage{IRTools.jl}):
\begin{quote}
  Using Cassette on code you wrote is a bit like shooting yourself with a experimental mind control
  weapon, to force your hands to move like you knew how to fly a helicopter.  Even if it works, you
  still had to learn to fly the helicopter in order to program the mind-control weapon to force
  yourself to act like you knew how to fly a helicopter.\footnote{Lyndon White (2020), private
    communication on \protect\url{https://julialang.slack.com}.}
\end{quote}

In conclusion, the dynamic graph tracking system of \irtrackerjl{} has turned out to be an
interesting idea with some application possibilities.  But the analysis of dynamic probabilistic
models could not be definitely solved.  In the course of development, though, various techniques
have been tried or ruled out, challenges identified, and other alternatives explored.  This
knowledge has lead me to a better understanding of the domain and some more advanced ideas for the
future, some of which are laid out in the following section.

\section{Future Work}
\label{sec:future-work}

While \irtrackerjl{} is quite a satisfying and complete system, the approach that \autogibbsjl{}
takes provides only an ad-hoc solution to a major shortcoming of \turingjl{}: the lack a structural
model representation that is open to analysis and transformations.  This has make me consider
alternatives, approaching the representation problem for probabilistic programming languages on a
more fundamental level.  Most of the following ideas I have already informally described
online\footnote{\protect\url{https://github.com/phipsgabler/probability-ir}}.

Let us review the important features of a universal, flexible PPL as mentioned in
section~\ref{sec:prob-prog}.  Its DSL should allow general recursion and nesting, support for all
language constructs and custom types and extensions, and be able to delegate to other samplers or
complex programs; and the internal representation should be such that multiple forms of analysis,
optimization, non-standard execution, and transformation can be performed.  Currently, \turingjl{}
is very primitive in this respect: one data structure (\jlinl{VarInfo}) contains a map from variable
names to values, the accumulated log-likelihood, and some other (mutable!) sampling
metadata. \autogibbsjl{}' solution consists of retrofitting some more structure onto this
representation~-- which is not ideal, and for proper analysis, it would be desirable to begin with a
better representation from the start, at least for the reasons mentioned above.

From difficulties described above, which became apparent during the implementation of the Gibbs
conditional extraction, together with the knowledge about \dppljl{}'s internals, I developed an
understanding of what an more advanced representation of probabilistic models, with a focus on
transformation and analysis, could be, from a metaprogramming and static analysis perspective,
employing ideas from functional programming language design.  The idealized goal would be for
variable names and dependency graphs in general probabilistic programs to behave nicely as abstract
data structures, and to be part of a closed, elegant, high-level language.  Many successful
approaches to PPL design probably come from the perspective of efficient and general inference
algorithm, putting the language design problem second to such a desire.  But it should be possible
to approach the field from a linguistic perspective as well.  A further goal would be to close the
gap between practical inference systems and the mostly theoretical, functional-programming-based
approaches of just formalizing probabilistic programs (such as probabilistic lambda calculi, or
type-theoretic formulations; see
\textcite{ramsey2002stochastic,heunen2017convenient,bhat2012type,scibior2015practical})

\newthought{Universal PPLs} have as their goal to let the user write down every model the language
allows, and still be able to do inference on it.  Of course, at the boundary of the space of
\enquote{reasonable} programs, trade-offs need to be made to still be able to do this.  It seem
advantageous to split up this conjunction: by creating a format in which one can denote every
possible model of a very large class, without a priori having to deal with the restrictions of
inference.  Then for each model, suitable transformations and analyses can be performed in a uniform
representation, and specialized backends be chosen from a wide range, which understand precisely the
fragment of the model language used.

What I propose is a \enquote{probabilistic intermediate representation}, that turns around how
things are currently construed in most of the approaches.  Instead of starting from a model as a
\enquote{sampling function}, which is evaluated to extract graphs or other symbolic representations
from it, one should begin from a representation that already is general, yet richly structured, and
derive inference programs from it.  Viewed from the opposite direction, in contrast to PPLs that are
built on top of a DSL representation, such an representation should be backend-agnostic, and instead
allow all kinds of models to be specified in a uniform syntax, without being constraint by the
demands of a specific sampling algorithm or inference technique.  And furthermore, it shouldn't
matter whether the model is complicated, nonparametric, dynamic~-- the object that is worked with is
always a fixed, full program in a specified syntax, with an intuitive denotation.

This separation between the a \enquote{specification abstraction} in form of a general
representation and \enquote{evaluator abstractions} provided by interfaces to multiple sampler
implementations seems novel.  The closest correspondence would be the formalization attempts of
probabilistic models through monads and type systems; but that is more semantic than syntactic.
There are some domain-specific \enquote{linguae frankae} like the syntax of Stan and JAGS, but they
are, too, somewhat restricted, and not independently defined and maintained~-- the systems coming
later just chose to take over the same kind of input format for their own implementation.
\juliapackage{Gen.jl} \parencite{cusumano-towner2020gen} provides an extensibel interface for the
class of models it supports, but this is still quite tightly bound to its inference system.  All
these approaches could rather be abstracted out into a model specification formalism in its own
right, that has more general analysis capabilities, and can then be transformed abstractly,
ultimately producing the form some concrete evaluator (i.e., sampling algorithm or PPL system)
requires.

The advantage of such aseparation, besides making available solutions and techniques from
programming language theory and compiler constrution, is that it provides a different kind of common
abstraction for PPLs than is possible through a \enquote{one DSL per system} approach.  Recently,
developers in Julia have started writing \enquote{bridge code} to allow PPL interaction: there is
invented a common interface that multiple PPL systems can be fit under, and then models in each can
be used from within the other at evaluation.  This is necessary due to the lack of division of each
system into an evaluator and a model specification part: they always go together.  (\dppljl{} is
itself supposed to define an extensible model description language, but in practice is still quite
strongly integrated with \turingjl{}.)

I believe that starting from a common model specification language is in many cases preferable, and
more general than just a common interface for evaluators.  Such interfaces tend to assume much more
about the internals, while the capabilities of universal probabilistic programs are essentially
fixed: the notation of random variables used in model specification by hand, extended through the
forms of an embedding programming language.  Starting from this, we could consider the following a
least upper bound of all the PPL modeling approaches:
\begin{itemize}
  \firmlist
\item General code: covered by normal Julia IR with SSA statements, branches, and blocks.
\item \enquote{Sampling statements}: special assignment forms for tildes, or assumptions and
  observations in \turingjl{} parlance, which relate names or values to distributions (or generally
  sub-models) in an declarative way.
\item First-class variable names: these may be quite complex, containing for example indexing,
  fields, link functions, etc., which can be identified and analyzed in a structured way.
\end{itemize}
Given this, it seems feasible to define arbitrary probabilistic programs an IR-like syntax, similar
to an extended SSA-form; the crucial point being that names and tildes are not separated from a host
language.  The idea amounts to writing out a directed graphical model with deterministic and
stochastic nodes, named random variables, but generalized to programs~-- e.g., allowing dynamic
structure with to branching and recursion.  A model in this kind of format then defines an abstract
and uninterpreted parametrized joint density function over its trace space (as given through the
unified name set of all possible runs, see e.g. \textcite{lew2020trace}), factorized into primitive
statements and blocks.

There is still much to be clarified and researched about the syntax and semantics of such a
representation, but the underlying principle should be intuitively clear by just matching the
existing Julia IR to probabilistic semantics of models like in \turingjl{}, \juliapackage{Soss.jl},
or \juliapackage{Gen.jl}.  Consider, for example, a hierarchical Gaussian model, informally written
as
\begin{lstlisting}
n = 1
while n <= N
    {x[n]} ~ Normal({mu[z[n]]})
    n += 1
end
{y} ~ MvNormal({x}, {sigma})
\end{lstlisting}
We can imagine this to be represented directly in probabilistic IR by conceiving of a lowering
mechanism that treats tilde statements just like assignments, and preserves variable names:
\begin{lstlisting}
1:
  goto 2 (1)
2 (%1):
  %2 = {z[%1]}
  %3 = {mu[%2]}
  {x[%1]} ~ Normal(%3)
  %4 = %1 < N
  br 4 unless %4
  br 3
3:
  %5 = %1 + 1
 br 2 (%5)
4:
  {y} ~ MvNormal({x}, {sigma})
\end{lstlisting}
If we define the tilde statements to behave like stochastic function calls, with a side effect of
somehow storing the intermediate stochastic values and their names as metadata, this is exactly how
the evaluation semantics of \dppljl{} in most cases work.

In contrast to \autogibbsjl{}'s data structures, this kind of model is not a slice, but preserves
the complete information about a model specification though a first-class representations.  The
probabilistic part of it, as opposed to the code generated by \dppljl{}, is referentially
transparent.  These properties make analysis and code transformations, similar to the ones possible
with \juliapackage{IRTools.jl}, significantly easier and more general.  On the formal representation
we can then apply transformations such as specialization on a constant parameter or observation,
resulting in a new model in IR form, exploitation of probabilistic knowledge, like collapsing or
conjugacy exploitation, or \enquote{disaggretating} a non-parametric model into something sampleable
(e.g., re-representing a Dirichlet process model with a CRP-based one), or other changes of the
probabilistic structure.  We can also apply static analysis or abstract interpretation techniques:
e.g., extraction of Gibbs conditionals.  Finally, the model can be converted into a form fit for
evalation~-- the transformation of the IR into executable code, or data structures for other PPL
systems.

\newthought{All these usages} can be represented through composition of several small structural
functions: constructing a new model that is the Gibbs conditional for some varible; turning a model
into a plain Julia generative function; extracting the log-likelihood function of the model;
specializing a model with some variable to given observations, and checking that the result is
static; perform a causal intervention on some random variable; of converting a model with fixed data
into a factor graph representation.

A similar principle of is currently developed with JAX \parencite{bradbury2018jax}, which is inteded
for numeric functions, and in which there exists a unified representation of functional programs
that undergo various transformations.  On top of JAX,
Oryx\footnote{\protect\url{https://www.tensorflow.org/probability/oryx}} should provide with the
necessary infrastructure to apply this in the setting of probabilistic programs.  JAX, though, is
closer to lambda calculus in A-normal form than SSA-form IR; it assumes referential transparency and
has no representation of control structures.  In Julia, Soss \parencite{scherrer2019soss} takes a
somewhat comparable approach by representing models written in a Julia DSL in completely symbolic
expression form, from which inference code is generated.  Also here, not the full generality of the
host language is available, but only a pure subset of it; and again, control structure can only be
realized through combinator functions, not at language level.

The approach most comparable to the presented ideas, although stemming from a complete different
domain, would be tactic or elaborator systems in proof assistants
\parencite[e.g.,][]{brady2013idris,coqdevelopmentteam2010coq}.  There, user-written programs are
iteratively refined into other, more specialized forms through functions expressed in a metalanguage
(the so-called tactics), interleaving automated transformations and manual interventions.  A similar
style of development could boost the usability and flexibility of Bayesian inference: after writing
a down model syntactically, the user can interactively refine the model code in symbolic form,
applying their knowledge and constraints, until they arrive at a form that can be passed to some
inference mechanism.

%%% Local Variables: 
%%% TeX-master: "main"
%%% End: