\chapter{Discussion}
\label{cha:discussion}

The history of this project forms a large arc, starting from a general problem in \turingjl{}, over
a digression into compiler technology and automatic differentiation, back the the implementation of
a proof of concept in the form of a very specific inference method.  As we have seen, two separate
pieces of software have emerged from it: \irtrackerjl{} and \autogibbsjl{}.  On its own, the latter
is indeed an improvement over the previous situation: Gibbs conditional samplers can be
significantly faster, thus allowing longer chains, than particle-based samplers, the go-to
instrument for discrete variables in \turingjl{} so far.  Already the addition of a \enquote{manual}
Gibbs conditional sampler in \turingjl{} allows to directly implement many models from the
literature, for which conditionals are often provided analytically.  Automatic derivation allows to
generalize this to a large class of models that have been found useful in other systems such as
JAGS.  The underlying issue~-- that \turingjl{} lacks a structural representation of models~-- is
not at all resolved by the implementation, unfortunately.  This makes \autogibbsjl{} not completly
satisfactory, since the recursion and branch tracking features of \irtrackerjl{} cannot be applied
in a useful way.

But the real difficulty is that dynamic models cannot be satisfactorily handled through
snapshot-like slices in the form of traces.  Systems trying to achieve this either become
restrictive in their expressibility, or very complex in some aspects, up to practical limitation
(see \textcite{mansinghka2014venture} and \textcite{goodman2012church}).  Furthermore, during
implementation of the work, the two main practical difficulties were matching of variable names
(e.g., subsuming \jlinl{x[1:10]} under \jlinl{x[1:3][2]}), and the correct handling of mutations
that shadow actual data dependencies (e.g., when one has an array \jlinl{x}, samples \jlinl{x[1]},
writes it to \jlinl{x} with \jlinl{setindex!}, and then uses \jlinl{getindex(x, i)} somewhere
downstream).  These cannot be really solved using external tooling, in the way that has been
demonstrated.  A more versatile dictionary structure for variable name keys could improve the
situation for variable names, but wouldn't solve all of the underlying issues.

There is also a fragility problem: Julia IR, while being publicly documented, is a rather internal
feature of the language, and may change between compiler versions.  The \juliapackage{IRTools.jl}
package provides a good mid-layer mitigating this, but still there's lot of reasons why a custom IR
would be nicer.  From the other side, also the structure of \dppljl{}'s model representations is not
stable, and actually an implementation detail that should not be relied on from the outside~--
especially not by an important feature such as dependency extration.  In a certain sense, the whole
approach is misguided: why rely on external tracking for a framework that is really under ones
control, using such heavy machinery as IR transformations?  This is illuminated by following very
telling quote about similar tendencies (\juliapackage{Cassette.jl} is a package very similar to
\juliapackage{IRTools.jl}):
\begin{quote}
  Using Cassette on code you wrote is a bit like shooting youself with a experimental mind control
  weapon, to force your hands to move like you knew how to fly a helicopter.  Even if it works, you
  still had to learn to fly the helicopter in order to program the mind-control weapon to force
  yourself to act like you knew how to fly a helicopter.\footnote{Lyndon White (2020), private
    communication on \protect\url{https://julialang.slack.com}.}
\end{quote}

In conclusion, the dynamic graph tracking system has turned out to be an interesting idea with some
application possibilities.  Unfortunatley, the analysis of dynamic probabilistic models could not be
shown to be among them.  In the course of development, though, various techniques have been tried or
ruled out, challenges identified, and other alternatives explored.  This knowledge has lead me to a
better understanding of the domain and some more advanced ideas for the future, some of which are
layed out in the following section.

\section{Future Work}
\label{sec:future-work}

While \irtrackerjl{} is quite a satisfying and complete system, the approach that \autogibbsjl{}
takes provides only an ad-hoc solution to one shortcoming of \turingjl{}: the lack a structural
model represenation that is open to analysis and transformations.  This has lead me to consider
alternatives, approaching the representation problem for probabilistic programming languages on a
more fundamental level.  Most of the following ideas I have already informally described
online\footnote{\protect\url{https://github.com/phipsgabler/probability-ir}}.

Let us review the important features of a universal, flexible PPL as mentioned in
section~\ref{sec:prob-prog}: its DSL should allow general recursion and nesting, support for all
language constructs and custom types and extensions, and be able to delegate to other samplers or
complex programs.  And the internal representation should be such that multiple forms of analysis,
optimization, non-standard execution, and transformation can be performed.  Currently, \turingjl{}
is very primitive in this respect: one data structure (\jlinl{VarInfo}) contains a map from variable
names to values, the accumulated log-likelihood, and some other (mutable!) sampling
metadata. \autogibbsjl{}' approach consists of retrofitting some more structure onto this
representation~-- this is not ideal, and for proper analysis, it would be desireable to begin with a
better representation from the start, at least for the reasons mentioned above.

From difficulties described above, which became apparent during the implementation of the Gibbs
conditional extraction, together with the knowledge about \dppljl{}'s internals, I developed an
understanding of what an more advanced representation of probabilistic models, with a focus on
transformation and analysis, could be, from a metaprogramming and static analysis perspective,
employing ideas from functional programming language design.  The idealized goal would be for
variable names and dependency graphs in general probabilistic programs to behave nicely as abstract
data structures, and to be part of a closed, elegant, high-level language.  Many successful
approaches to PPL design probably come from the perspective of efficient and general inference
algorithm, putting the language design problem second to such a desire.  But it should be possible
to approach the field from a linguistic perspective as well.  A further goal would be to close the
gap between practical inference systems and the mostly theoretical, functional-programming-based
approaches of just formalizing probabilistic programs (such as probabilistic lambda calculi, or
type-theoretic formulations; see
\textcite{ramsey2002stochastic,heunen2017convenient,bhat2012type,scibior2015practical})

\newthought{Universal PPLs} have as their goal to let the user write down every model the language
allows, and still be able to do inference on it.  Of course, at the boundary of the space of
reasonable programs, tradeoffs need to be made to still be able to do this.  It seem advantageous to
split up this conjuction: by creating a format in which one can denote every possible model of a
very large class, without a priori having to deal with the restrictions of inference.  Then for each
model, suitable transformations and analyses can be performd in a uniform representation, and
specialized backends be chosen from a wide range, which understand precisely the fragment of the
model language used.

What I propose is a \enquote{probabilistic intermediate representation}, that in a way turns around
how things are currently construed in most of the approaches.  Instead of starting from a model as a
\enquote{sampling function}, which is evaluated to extract graphs or other symbolic representations
from it, one should start from a representation that already is general, yet richly structured, and
derive evaluators from it.  Viewed from the opposite direction, in contrast to PPLs that are built
on top of a DLS representation, such an representation should be backend-agnostic, and instead allow
all kinds of models to be specified in a uniform syntax, without being constraint by the demands of
a specific sampling algorithm or inference technique.  And if furthermore it shouldn't matter
whether the model is complicated, nonparametric, dynamic~-- the object that is worked with is always
a fixed, full program in a specified syntax, with an intuitive denotation.

This separation between the a \enquote{specification abstraction} in form of a general
representation and \enquote{evaluator abstractions} provided by interfaces to multiple sampler
implementations seems novel.  The closest correspondence would be the formalization attempts of
probabilistic models through monads and type systems; but that is more semantic than syntactic.
There are some domain-specific \enquote{linguae frankae} like the syntax of Stan and JAGS, but they
are, too, somewhat restricted, and not independently defined and maintained~-- the systems coming
later just chose to take over the same kind of input format for their own implemenation.  All these
approaches could rather be abstracted out into a model specification formalism in its own right,
that has more general analysis capabilites, and can then be transformed abstractly, ultimately
producing the format some concrete evaluator (i.e., sampling algorithm) requires.

The advantage of this kind of approach, besides making available solutions and techniques from
programming language theory, is that it provides a different kind of common abstraction for PPLs
than is possible through a DSL per system.  Recently, people have started writing \enquote{bridge
  code} to allow PPL interaction: there is invented a common interface that multiple PPL systems can
be fit under, and then models in each can be used from within the other at evaluation.  This is
necessary due to the lack of division of a system into an evalator and a model specification part:
they always go together.  I believe that starting from a common model specification language is in
many cases preferrable, and more general than defining a common interface for evaluators.

The latter tends to assume much more about the internals, while the capabilities of universal
probabilistic programs are essentially fixed: the notation of random variables used in model
specification by hand, extended through the forms of an embedding programming language.  Such a sort
of least upper bound of all the PPL modelling approaches, consists of
\begin{itemize}
  \firmlist
\item General Julia code: this can most conveniently be represented like normal Julia IR with SSA
  statements, branches, and blocks.
\item \enquote{Sampling statements}: tildes, or assumptions and observations in \turingjl{}
  parlance, which relate names or values to distributions in an immutable way.
\item Variable names: these may be quite complex, containing for example indexing, fields, link
  functions, etc., which can be identified and analyzed in a structured way.
\end{itemize}
Given this, it seems feasible to define arbitrary probabilistic programs an IR-like syntax, similar
to an extended SSA-form; the crucial point being that names and tildes are not separated from a host
language.  This amounts to writing out a directed graphical model with deterministic and stochastic
nodes, named random variables, but generalized to programs~-- e.g., allowing dynamic structure due
to branching and recursion.  A model in this kind of format then defines an abstract and
uninterpreted parametrized joint density function over its trace space (as given through the unified
name set of all possible runs), factorized into primitive statements and blocks.\todo{wrap up \& examples}

A similar principle is currently developed with JAX \parencite{bradbury2018jax}, in which there
exists a unified representation of functional programs that undergo various transformations, and
which Oryx\footnote{\protect\url{https://www.tensorflow.org/probability/oryx}} should provide with
the necessary infrastructure to apply this in the setting of probabilistic programs.  JAX, though,
is closer to lambda calculus in A-normal form than SSA-form IR; it assumes referential transparency
and has no representation of control structures.  In Julia, Soss \parencite{scherrer2019soss} takes
a somewhat comparable approach by representing models written in a Julia DSL in completely symbolic
expression form, from which inference code is generated.  Also here, not the full generality of the
host language is available, but only a pure subset of it; and again, constrol structure can only be
realized through combinator functions, not at language level.

Most comparable, although stemming from a complete different domain, would be tactic or elaborator
systems in proof assistents \parencite[e.g.,][]{brady2013idris,coqdevelopmentteam2010coq}.  There,
user-written programs are iteratively refined into other, more specialized forms through functions
expressed in a metalanguage (the so-called tactics), interleaving automated transformations and
manual interventions.  A similar style of development could profit the usability of Bayesian
inference: after writing a down model syntactically, the user can interactively refine the model
code in symbolic form, applying their knowledge and constraints, until they arrive at a form that
can be passed to some inference evaluator.

%%% Local Variables: 
%%% TeX-master: "main"
%%% End: