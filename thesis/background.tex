\chapter{Background}
\label{cha:background}

This section provides the background for the concepts used later in
chapters~\ref{cha:impl-dynam-graph} and \ref{cha:graph-track-prob}.  On the one hand, it gives a
quick overview of Baysian inference and probabilistic programming in general, necessary to
understand the requirements and usual approaches of probabilitic programming systems.

On the other hand, the machinery and language used to develop the graph tracking system forming the
main part of the work are described.  This consists firstly of a short introduction to graph
tracking and source-to-source automatic differentiation, which contain many ideas and terminology
that will be used later, and often provided inspiration.  Second, the basic notions and techniques
of the Julia compilation process, as well as the language's metaprogramming capabilities are
described, which form the basis of the implementation.


%%%%%%%%%%%%%%%%%%%%%%%%%%%%%%%%%%%%%%%%%%%%%%%%%%%%%%%%%%%%%%%%%%%%%%%%%%%%%%%%%%%%%%%%%%%%%%%%%%%
\section{Bayesian Inference and Probabilistic Programming}
\label{sec:bayes-infer-prob}

Probabilistic modelling is a technique for modelling phenomena based on the assumption that
observables can be fully described through some stochastic process.  When we assume this process to
belong to a certain family of processes, the estimation of the \enquote{best} process is a form of
learning: if we have a good description of how obserations are generated, we can make summary
statements about the whole population (descriptive statistics) or make predictions about new
observations.  When observations come in pairs

Within a Baysian statistical framework, we assume that the family of processes that is used


bla\footnote{Note the abuse of notation with \(\prob{}\) and the integral; see~\ref{cha:notation}.}
\begin{equation}
  \prob{\theta \given x} = \frac{\prob{x \given \theta}\; \prob{\Theta}}{\int \prob{x \given
    t} \dif t}
\end{equation}

This kind of learning is called Bayesian inference.

When the distributions involved form a sufficiently \enquote{nice} combination, e.g., when they form
a conjugate pair, the posterior distribution can be calculated analytically.  In general, however,
this


\subsection{Probabilistic Programming}
\label{sec:prob-prog}

Probabilistic programming is a means of describing probabilistic models through the syntax of a
programming language. Probabilistic programms distinguish themselves from normal programs by the
possibility of being sampled from conditionally, with some of the internal variables fixed to
observed values.  While probabilistic programming systems are often implemented as separate,
domain-specific languages, they can also be embedded into \enquote{host} programming languages with
sufficient syntactic flexibility.  The latter is advantageous if one wants to use regular
general-purpose programming constructs or interact with other functionalities of the host language.




%%%%%%%%%%%%%%%%%%%%%%%%%%%%%%%%%%%%%%%%%%%%%%%%%%%%%%%%%%%%%%%%%%%%%%%%%%%%%%%%%%%%%%%%%%%%%%%%%%%
\section{Computation Graphs and Automatic Differentiation}
\label{sec:graph-track-autom}


%%%%%%%%%%%%%%%%%%%%%%%%%%%%%%%%%%%%%%%%%%%%%%%%%%%%%%%%%%%%%%%%%%%%%%%%%%%%%%%%%%%%%%%%%%%%%%%%%%%
\section{Metaprogramming and Compilation in Julia}
\label{sec:metapr-comp-julia}




%%% Local Variables: 
%%% TeX-master: "main"
%%% End: